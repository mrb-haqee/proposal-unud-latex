\chapter*{BAB I\\PENDAHULUAN}
\stepcounter{chapter}
\addcontentsline{toc}{chapter}{BAB I. PENDAHULUAN}

\section{Latar Belakang}
Untuk sitasi seperti ini \cite{Luschi2013}, ini \citep{Luschi2013}, atau ini \citet{Luschi2013}

\noindent text tanpa indentasi \blindtext[1]

membuat list nomor
\begin{enumerate}[noitemsep]
    \item Item Pertama
    \item Item Kedua
    \item Item Ketiga
\end{enumerate}

membuat list nomor
\begin{enumerate}[label=\alph*]
    \item Item Pertama
    \item Item Kedua
    \item Item Ketiga
\end{enumerate}

Gambar
\begin{center}
    \includegraphics[width=10cm]{logo-unud.png}
    \captionof{figure}{Diagram Model dengan dua Kompartemen}
    \label{fig:2-kompartemen}
\end{center}

perumusan
\begin{equation}\label{eq:rumus-1}
    \begin{aligned}
        \frac{dy(t)}{dt} & = \beta x(t)y(t) - \eta y(t)
    \end{aligned}
\end{equation}

refrensi rumus di atas \eqref{eq:rumus-1}

tabel
\begin{longtable}{ll}
    \caption{Keterangan parameter pada model}                   \\
    \toprule
    \textbf{Variabel}  & \textbf{Deskripsi}                     \\
    \midrule
    $x$                & Jumlah Penyu                           \\
    $y$                & Jumlah Manusia yang mengkonsumsi penyu \\
    \midrule
    \textbf{Parameter} & \textbf{Deskripsi}                     \\
    \midrule
    $r$                & Laju pertumbuhan populasi penyu        \\
    $\alpha$           & Tingkat konsumsi oleh manusia          \\
    $s$                & Tingkat pertumbuhan populasi manusia   \\
    $\omega$           & Laju kematian alami penyu              \\
    \bottomrule
\end{longtable}

\section{Rumusan Masalah}
\section{Batasan Masalah}
\section{Tujuan Penelitian}
\section{Manfaat Penelitian}