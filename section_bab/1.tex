\chapter*{
  BAB I\\
  \vspace{-10pt}
  PENDAHULUAN
 }

\stepcounter{chapter}
\addcontentsline{toc}{chapter}{BAB I. PENDAHULUAN}

\section{Latar Belakang}
Panduan latex $\rightarrow$ \href{https://www.overleaf.com/learn/latex/Learn_LaTeX_in_30_minutes}{www.overleaf.com}
\section{Indentasi}
Text dengan indentasi \lipsum[1][1-3]

\noindent Text tanpa indentasi \lipsum[1][1-3]
\textit{italic}, \textbf{bold}, \textbf{\textit{italic-bold}}

\section{Sitasi}
\noindent Terdapat beberapa penulisan sitasi, yaitu:
\begin{enumerate}
    \item Seperti ini \parencite{Luschi2013} untuk (Author year)
    \item Seperti ini \textcite{Luschi2013} untuk Author (year)
\end{enumerate}


\section{Perumusan}
\noindent Perumusan inline diapit oleh tanda \$ untuk membuat rumus, contoh $\dot{x}=\frac{df}{dx}$.
\noindent Perumusan tanpa label dan di paragraf baru diapit menggunakan \$\$, contoh
$$
    \frac{dy(t)}{dt} = \beta x(t)y(t) - \eta y(t)
$$

\noindent Selanjutnya menggunakan 'equation' untuk penomoran rumus
\begin{equation}\label{eq:1}
    \frac{dy(t)}{dt} = \beta x(t)y(t) - \eta y(t)
\end{equation}
\noindent refrensi rumus di atas \eqref{eq:1}

\section{Numbering}
\noindent Membuat list angka
\begin{enumerate}
    \item Item Pertama
    \item Item Kedua
    \item Item Ketiga
\end{enumerate}
\noindent Membuat list huruf
\begin{enumerate}[label=\alph*]
    \item Item Pertama
    \item Item Kedua
    \item Item Ketiga
\end{enumerate}
\noindent Membuat list simbol
\begin{itemize}
    \item Item Pertama
    \item Item Kedua
    \item Item Ketiga
\end{itemize}

\section{Menambahkan Gambar}
\begin{figure}[ht!]
    \label{fig:1}
    \centering
    \includegraphics[width=0.2\textwidth]{logo-unud.png}
    \caption{logo unud}
\end{figure}
\begin{center}
    \includegraphics[width=0.2\textwidth]{logo-unud.png}
    \captionof{figure}{logo unud}
    \label{fig:2}
\end{center}



\section{Tabel}
Gunakan table atau longtable, bedanya longtable dapat dipecah seperti dibawah ini
\begin{longtable}{ll}
    \caption{caption tabel}                   \\
    \toprule
    \textbf{Variabel}  & \textbf{Deskripsi}   \\
    \midrule
    $x$                & Variable X           \\
    $y$                & Variable Y           \\
    \midrule
    \textbf{Parameter} & \textbf{Deskripsi}   \\
    \midrule
    $r$                & Parameter $r$        \\
    $\alpha$           & Parameter $\alpha$   \\
    $s$                & Parameter  $s$       \\
    $\omega$           & Parameter   $\omega$ \\
    \bottomrule
\end{longtable}
\noindent atau tabel biasa
\begin{table}[h!]
    \centering
    \caption{caption tabel}
    \begin{tabular}{|c|c|}
        \hline
        \textbf{Variabel}  & \textbf{Deskripsi}   \\
        \hline
        $x$                & Variable X           \\
        $y$                & Variable Y           \\
        \hline
        \textbf{Parameter} & \textbf{Deskripsi}   \\
        \hline
        $r$                & Parameter $r$        \\
        $\alpha$           & Parameter $\alpha$   \\
        $s$                & Parameter  $s$       \\
        $\omega$           & Parameter   $\omega$ \\
        \hline
    \end{tabular}
\end{table}

\section{Rumusan Masalah}
\section{Batasan Masalah}
\section{Tujuan Penelitian}
\section{Manfaat Penelitian}