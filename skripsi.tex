\documentclass[a4paper, 12pt]{report}
\usepackage[left=4cm, top=4cm, right=3cm, bottom=3cm]{geometry}
\usepackage[utf8]{inputenc}
\usepackage{ragged2e} % Paket untuk penjustifikasi

\usepackage{graphicx} % libarary gambar
\graphicspath{{images/}}

\usepackage{amsmath,amsfonts,amssymb, mathptmx} %library matematika
\usepackage{multicol, booktabs, longtable} %library tabel
\usepackage{caption}

\usepackage{blindtext} %dummy text
\usepackage{titlesec} % Library untuk Pengaturan format judul
\usepackage{indentfirst} % Library untuk Pengaturan Indentasi
\usepackage{fancyhdr} % Library untuk Pengaturan kepala (header) dan kaki (footer)
\usepackage{setspace} % Library untuk pengaturan spacing Line
\usepackage{natbib} %library untuk sitasi
\usepackage{enumitem} % Library untuk mengatur enumerate

\usepackage{hyperref}
\usepackage[all]{hypcap}
\captionsetup[figure]{hypcap=true}

% pengaturan pemisahan kata
\pretolerance=10000 % Mengurangi pemisahan kata
\hyphenpenalty=10000 % Mengurangi pemisahan kata

% Merubah nama chapter
\renewcommand{\contentsname}{DAFTAR ISI}
\renewcommand{\chaptername}{BAB}
\renewcommand{\bibname}{DAFTAR PUSTAKA}
\renewcommand{\listfigurename}{DAFTAR GAMBAR}
\renewcommand{\listtablename}{DAFTAR TABEL}
\renewcommand{\figurename}{Gambar}
\renewcommand{\listtablename}{DAFTAR TABEL}
\renewcommand{\tablename}{Tabel}
\renewcommand{\appendixname}{Lampiran}

% Setting angka di dokumen
\sloppy
\renewcommand{\thechapter}{\Roman{chapter}}
\renewcommand{\thesection}{\arabic{chapter}.\arabic{section}}
\renewcommand{\thesubsection}{\arabic{chapter}.\arabic{section}.\arabic{subsection}}
\renewcommand{\thefigure}{\arabic{chapter}.\arabic{figure}}
\renewcommand{\thetable}{\arabic{chapter}.\arabic{table}}
\renewcommand{\theequation}{\arabic{chapter}.\arabic{equation}}
\setcounter{secnumdepth}{3}
\renewcommand{\thesubsubsection}{\thesubsection\arabic{subsubsection}}

% Mengatur size dokumen
\titleformat{\chapter}[display]{\centering\normalfont\large\bfseries}{\MakeUppercase{\chaptertitlename}\ \thechapter}{0pt}{\large}
\titlespacing*{\chapter}{0pt}{-50pt}{40pt}
\titleformat{\section}{\normalfont\fontsize{12}{15}\bfseries}{\thesection}{1em}{}
\titleformat{\subsection}{\normalfont\fontsize{12}{15}\bfseries}{\thesubsection}{1em}{}

% penggunaan first line % gunakan \noindent untuk normal line
\setlength\parindent{36pt}

% Mebgatur gaya dokumen
\pagestyle{empty}
\renewcommand{\headrulewidth}{0pt}

% Library untuk daftar isi, gambar, dan tabel
\usepackage{tocloft}
\setlength{\cftbeforechapskip}{1pt} % Jarak sebelum bab
\setlength{\cftbeforesecskip}{1pt}  % Jarak sebelum subbab
\setlength{\cftbeforesubsecskip}{1pt} % Jarak sebelum sub-subbab
\renewcommand{\cftdotsep}{1}
% Pengaturan judul daftar isi, daftar tabel, dan daftar gambar
\renewcommand{\cfttoctitlefont}{\hfil\bfseries\normalsize} % Daftar Isi
\renewcommand{\cftloftitlefont}{\hfill\bfseries\normalsize} % Daftar Tabel
\renewcommand{\cftlottitlefont}{\hfill\bfseries\normalsize} % Daftar Gambar
\renewcommand{\cftaftertoctitle}{\hfil} % Posisi setelah judul Daftar Isi
\renewcommand{\cftafterloftitle}{\hfill} % Posisi setelah judul Daftar Tabel
\renewcommand{\cftafterlottitle}{\hfill} % Posisi setelah judul Daftar Gambar
\setlength{\cftbeforetoctitleskip}{0pt} % Jarak sebelum judul Daftar Isi


% Variable
\newcommand{\Judul}{Tulis Judul disini Sesuai Topik}
\newcommand{\Nama}{Muhammad Rafli Baihaqi}
\newcommand{\NIM}{2008541999}
\newcommand{\Kompetensi}{Komputasi}

\newcommand{\Tanggal}{23 Januari 2024} % tanggal penulisan di kata pengantar
\newcommand{\Seminar}{-} %tanggal seminar

%====================START PENULISAN==========================
\begin{document}
\justifying

% line spacing
\setstretch{1.5}

% COVER SAMPUL
\begin{titlepage}
    \centering
    \textbf{PROPOSAL TUGAS AKHIR}\\\vfill
    \textbf{\MakeUppercase{\Judul}}\\\vfill
    \textbf{KOMPETENSI \MakeUppercase{\Kompetensi}}\vfill
    \includegraphics[width=4cm]{logo-unud.png}\vfill
    \textbf{\MakeUppercase{\Nama}}\\
    \textbf{\NIM}\vfill
    \textbf{
        PROGRAM STUDI MATEMATIKA\\
        FAKULTAS MATEMATIKA DAN ILMU PENGETAHUAN ALAM\\
        UNIVERSITAS UDAYANA\\
        BUKIT JIMBARAN\\
        \the\year
    }\\\vfill
\end{titlepage}
\clearpage

\pagenumbering{roman}

% COVER HALAMAN
\chapter*{\centering\normalsize\textbf{\MakeUppercase{\Judul}}}
\addcontentsline{toc}{chapter}{LEMBAR JUDUL}
\begin{center}
    \textbf{KOMPETENSI \MakeUppercase{\Kompetensi}}\vfill
    \includegraphics[width=4cm]{logo-unud.png}\vfill
    \textbf{\MakeUppercase{\Nama}}\\
    \textbf{\NIM}\vfill
    \textbf{
        PROGRAM STUDI MATEMATIKA\\
        FAKULTAS MATEMATIKA DAN ILMU PENGETAHUAN ALAM\\
        UNIVERSITAS UDAYANA\\
        BUKIT JIMBARAN\\
        \the\year
    }\\\vfill
\end{center}




\clearpage


\chapter*{\centering\small LEMBAR PENGASAHAN PROPOSAL TUGAS AKHIR}
\addcontentsline{toc}{chapter}{LEMBAR PENGESAHAN}
\setstretch{1.5}
\noindent \begin{tabular}{p{3.3cm}p{9.5cm}}
  Judul \hspace{4.7em}:           & \Judul    \\
  Kompetensi \hspace{2.17em}:     & \Kompetensi \\
  Nama \hspace{4.58em}:           & \Nama     \\
  NIM \hspace{5em}:               & \NIM      \\
  Tanggal Seminar \hspace{0.2em}: & \Seminar  \\
\end{tabular}

\vspace{1cm}

\begin{center}
  Disetujui Oleh: \\
  \begin{multicols}{2}
    {Pembimbing II\\
      \vspace{2.75cm}
      \underline{I Putu Winada Gautama, S.Si., M.Sc.}\\
      NIP. 1991052820181113002\\}

    {Pembimbing I\\
      \vspace{2.75cm}
      \underline{Dr. Drs. G.K. Gandhiadi, M.T.} \\
      NIP. 196209301988031002\\}
  \end{multicols}

  \vspace{1cm}

  Mengetahui, \\
  Komisi Tugas Akhir \\
  Program Studi Matematika FMIPA Unud \\
  Ketua \\
  \vspace{2.75cm}
  \underline{I Wayan Sumarjaya, S.Si., M.Stats} \\
  NIP. 	197704212005011001 \\
\end{center}
\clearpage

\input{header/kata-pengantar.tex}
\clearpage

\tableofcontents % Daftar isi utama
\addcontentsline{toc}{chapter}{DAFTAR ISI}
\clearpage

\listoffigures % Daftar isi utama
\addcontentsline{toc}{chapter}{DAFTAR GAMBAR}
\clearpage

\listoftables % Daftar isi utama
\addcontentsline{toc}{chapter}{DAFTAR TABEL}
\clearpage

\pagenumbering{arabic}

\setstretch{2.0}

\chapter*{BAB I\\PENDAHULUAN}
\stepcounter{chapter}
\addcontentsline{toc}{chapter}{BAB I. PENDAHULUAN}

\section{Latar Belakang}
Untuk sitasi seperti ini \cite{Luschi2013}, ini \citep{Luschi2013}, atau ini \citet{Luschi2013}

\noindent text tanpa indentasi \blindtext[1]

membuat list nomor
\begin{enumerate}[noitemsep]
    \item Item Pertama
    \item Item Kedua
    \item Item Ketiga
\end{enumerate}

membuat list nomor
\begin{enumerate}[label=\alph*]
    \item Item Pertama
    \item Item Kedua
    \item Item Ketiga
\end{enumerate}

Gambar
\begin{center}
    \includegraphics[width=10cm]{logo-unud.png}
    \captionof{figure}{Diagram Model dengan dua Kompartemen}
    \label{fig:2-kompartemen}
\end{center}

perumusan
\begin{equation}\label{eq:rumus-1}
    \begin{aligned}
        \frac{dy(t)}{dt} & = \beta x(t)y(t) - \eta y(t)
    \end{aligned}
\end{equation}

refrensi rumus di atas \eqref{eq:rumus-1}

tabel
\begin{longtable}{ll}
    \caption{Keterangan parameter pada model}                   \\
    \toprule
    \textbf{Variabel}  & \textbf{Deskripsi}                     \\
    \midrule
    $x$                & Jumlah Penyu                           \\
    $y$                & Jumlah Manusia yang mengkonsumsi penyu \\
    \midrule
    \textbf{Parameter} & \textbf{Deskripsi}                     \\
    \midrule
    $r$                & Laju pertumbuhan populasi penyu        \\
    $\alpha$           & Tingkat konsumsi oleh manusia          \\
    $s$                & Tingkat pertumbuhan populasi manusia   \\
    $\omega$           & Laju kematian alami penyu              \\
    \bottomrule
\end{longtable}

\section{Rumusan Masalah}
\section{Batasan Masalah}
\section{Tujuan Penelitian}
\section{Manfaat Penelitian}
\clearpage

\chapter*{BAB II\\TINJAUAN PUSTAKA}
\thispagestyle{chapterfirstpage}
\stepcounter{chapter}
\addcontentsline{toc}{chapter}{BAB II. TINJAUAN PUSTAKA}

\section{Penelitian Sebelumnya}
\blindtext
\section{Literatur Metode}
\clearpage

\chapter*{BAB III\\METODOLOGI PENELITIAN}
\stepcounter{chapter}
\addcontentsline{toc}{chapter}{BAB III. METODOLOGI PENELITIAN}

\section{Tempat, Waktu, dan Data Penelitian}
\blindtext

\section{Analisis Metode}
\section{Teknik Penyeleaian Masalah}
\clearpage

\setstretch{1.5}
\bibliographystyle{apalike}
\bibliography{library}
\addcontentsline{toc}{chapter}{DAFTAR PUSTAKA}

\end{document}