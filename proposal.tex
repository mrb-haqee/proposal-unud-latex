\documentclass[a4paper, 12pt, oneside]{report}
% \usepackage{showframe} % memunculkan frame

% usepackage/library
\input{preambel/config.tex}
\addbibresource{referensi.bib}

% Variable
\newcommand\Judul{Pemodelan Matematika dalam Mencegah Kepunahan Populasi Penyu di Bali}
\newcommand\Nama{Muhammad Rafli Baihaqi}
\newcommand\NIM{2008541066}
\newcommand\Kompetensi{Komputasi}

\newcommand\Tanggal{31 Januari 2024} % tanggal penulisan di kata pengantar
\newcommand\Seminar{7 Februari 2024} % tanggal seminar

\newarray\dospemsatu
\readarray{dospemsatu}{Dr. Drs. G.K. Gandhiadi, M.T.&196209301988031002} % {nama+gelar&nip}
\newarray\dospemdua
\readarray{dospemdua}{I Putu Winada Gautama, S.Si., M.Sc.&1991052820181113002} % {nama+gelar&nip}

%====================START PENULISAN==========================
\begin{document}
\justifying

\setstretch{1.5} % line spacing

% COVER SAMPUL
\input{section_header/1_cover.tex}
\clearpage

\pagenumbering{roman}

% COVER HALAMAN
\input{section_header/2_cover_hal.tex}
\clearpage

\input{section_header/3_pengesahan.tex}
\clearpage

\chapter*{\centering\small KATA PENGANTAR}
\addcontentsline{toc}{chapter}{KATA PENGANTAR}

Puji dan syukur penulis panjatkan ke hadirat Tuhan Yang Maha Esa atas segala limpahan rahmat, berkat, kasih karunia, dan bimbingan-Nya sehingga penulis dapat menyelesaikan Proposal Tugas Akhir ini dengan judul \textbf{"\Judul"}. Penulisan Proposal Tugas Akhir ini tidak lepas dari bantuan, saran, bimbingan dan arahan dari berbagai pihak. Oleh karena itu, pada kesempatan ini penulis menyampaikan rasa terima kasih kepada:

\begin{enumerate}
  \item I Gusti Ayu Made Srinadi, S.Si., M.Si., selaku Ketua Jurusan Matematika FMIPA Universitas Udayana.
  \item I Wayan Sumarjaya, S.Si., M.Stats., selaku Ketua Komisi Tugas Akhir Jurusan Matematika.
  \item Dr. Drs. G.K. Gandhiadi, M.T., selaku Pembimbing I yang telah banyak memberikan bimbingan selama proses penulisan Tugas Akhir ini.
  \item I Putu Winada Gautama, S.Si., M.Sc., selaku Pembimbing II yang senantiasa membantu penulis selama proses penulisan Tugas Akhir ini.
\end{enumerate}

Saya menyadari bahwa masih banyak terdapat kekurangan dalam penyusunan Proposal ini karena keterbatasan pengetahuan dan pengalaman. Untuk itu saya sangat mengharapkan kritik dan saran yang membangun dari pembaca.

\begin{flushright}
  Bukit Jimbaran, \Tanggal\\
  \vspace{2cm}
  Penulis
\end{flushright}
\clearpage

\tableofcontents % Daftar isi utama
\addcontentsline{toc}{chapter}{DAFTAR ISI}
\clearpage

\listoffigures % Daftar isi utama
\addcontentsline{toc}{chapter}{DAFTAR GAMBAR}
\clearpage

\listoftables % Daftar isi utama
\addcontentsline{toc}{chapter}{DAFTAR TABEL}
\clearpage

\pagenumbering{arabic}

\setstretch{2.0}
\pagestyle{otherpages}

\chapter*{
  BAB I\\
  \vspace{-10pt}
  PENDAHULUAN
 }

\stepcounter{chapter}
\addcontentsline{toc}{chapter}{BAB I. PENDAHULUAN}

\section{Latar Belakang}
Panduan latex $\rightarrow$ \href{https://www.overleaf.com/learn/latex/Learn_LaTeX_in_30_minutes}{www.overleaf.com}
\section{Indentasi}
Text dengan indentasi \lipsum[1][1-3]

\noindent Text tanpa indentasi \lipsum[1][1-3]
\textit{italic}, \textbf{bold}, \textbf{\textit{italic-bold}}

\section{Sitasi}
\noindent Terdapat beberapa penulisan sitasi, yaitu:
\begin{enumerate}
    \item Seperti ini \parencite{Luschi2013} untuk (Author year)
    \item Seperti ini \textcite{Luschi2013} untuk Author (year)
\end{enumerate}


\section{Perumusan}
\noindent Perumusan inline diapit oleh tanda \$ untuk membuat rumus, contoh $\dot{x}=\frac{df}{dx}$.
\noindent Perumusan tanpa label dan di paragraf baru diapit menggunakan \$\$, contoh
$$
    \frac{dy(t)}{dt} = \beta x(t)y(t) - \eta y(t)
$$

\noindent Selanjutnya menggunakan 'equation' untuk penomoran rumus
\begin{equation}\label{eq:1}
    \frac{dy(t)}{dt} = \beta x(t)y(t) - \eta y(t)
\end{equation}
\noindent refrensi rumus di atas \eqref{eq:1}

\section{Numbering}
\noindent Membuat list angka
\begin{enumerate}
    \item Item Pertama
    \item Item Kedua
    \item Item Ketiga
\end{enumerate}
\noindent Membuat list huruf
\begin{enumerate}[label=\alph*]
    \item Item Pertama
    \item Item Kedua
    \item Item Ketiga
\end{enumerate}
\noindent Membuat list simbol
\begin{itemize}
    \item Item Pertama
    \item Item Kedua
    \item Item Ketiga
\end{itemize}

\section{Menambahkan Gambar}
\begin{figure}[ht!]
    \label{fig:1}
    \centering
    \includegraphics[width=0.2\textwidth]{logo-unud.png}
    \caption{logo unud}
\end{figure}
\begin{center}
    \includegraphics[width=0.2\textwidth]{logo-unud.png}
    \captionof{figure}{logo unud}
    \label{fig:2}
\end{center}



\section{Tabel}
Gunakan table atau longtable, bedanya longtable dapat dipecah seperti dibawah ini
\begin{longtable}{ll}
    \caption{caption tabel}                   \\
    \toprule
    \textbf{Variabel}  & \textbf{Deskripsi}   \\
    \midrule
    $x$                & Variable X           \\
    $y$                & Variable Y           \\
    \midrule
    \textbf{Parameter} & \textbf{Deskripsi}   \\
    \midrule
    $r$                & Parameter $r$        \\
    $\alpha$           & Parameter $\alpha$   \\
    $s$                & Parameter  $s$       \\
    $\omega$           & Parameter   $\omega$ \\
    \bottomrule
\end{longtable}
\noindent atau tabel biasa
\begin{table}[h!]
    \centering
    \caption{caption tabel}
    \begin{tabular}{|c|c|}
        \hline
        \textbf{Variabel}  & \textbf{Deskripsi}   \\
        \hline
        $x$                & Variable X           \\
        $y$                & Variable Y           \\
        \hline
        \textbf{Parameter} & \textbf{Deskripsi}   \\
        \hline
        $r$                & Parameter $r$        \\
        $\alpha$           & Parameter $\alpha$   \\
        $s$                & Parameter  $s$       \\
        $\omega$           & Parameter   $\omega$ \\
        \hline
    \end{tabular}
\end{table}

\section{Rumusan Masalah}
\section{Batasan Masalah}
\section{Tujuan Penelitian}
\section{Manfaat Penelitian}
\clearpage

\chapter*{
    BAB II\\
    \vspace{-10pt}
    TINJAUAN PUSTAKA
}
\stepcounter{chapter}
\addcontentsline{toc}{chapter}{BAB II. TINJAUAN PUSTAKA}

\section{Penelitian Sebelumnya}
\blindtext
\section{Literatur Metode}
\clearpage

\chapter*{BAB III\\METODOLOGI PENELITIAN}
\stepcounter{chapter}
\addcontentsline{toc}{chapter}{BAB III. METODOLOGI PENELITIAN}

\section{Tempat, Waktu, dan Data Penelitian}
\blindtext

\section{Analisis Metode}
\section{Teknik Penyeleaian Masalah}
\clearpage

\setstretch{1.5}
\printbibliography
\end{document}