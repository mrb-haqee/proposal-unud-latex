\usepackage[utf8]{inputenc}

\usepackage[left=4cm, top=4cm, right=3cm, bottom=3cm]{geometry} % setting margin paper
\usepackage{times} % font Times New Roman
\usepackage{ragged2e} % setting paragraf style justify

\usepackage{graphicx} % library gambar
\graphicspath{{images/}}

\usepackage{amsmath,amsfonts,amssymb, mathptmx} %library matematika
\usepackage{multicol, booktabs, longtable} %library tabel
\usepackage{caption}
\usepackage{hyperref}
\usepackage[all]{hypcap}
\captionsetup{labelsep=space, hypcap=true}

\usepackage{arrayjobx} %membuat variable array
\usepackage{lipsum, blindtext} % dummy text
\usepackage{titlesec} % Library untuk Pengaturan format judul
\usepackage{indentfirst} % Library untuk Pengaturan Indentasi
\usepackage{fancyhdr} % Library untuk Pengaturan kepala (header) dan kaki (footer)
\usepackage[nodisplayskipstretch]{setspace} % Library untuk pengaturan spacing Line
\usepackage{enumitem} % Library untuk mengatur enumerate
\setlist[enumerate,itemize]{noitemsep,nosep}

% Library Referensi
\usepackage[style=apa,backend=biber]{biblatex}%maxcitenames=1

\usepackage{adjustbox}

% Merubah nama chapter
\renewcommand\contentsname{DAFTAR ISI}
\renewcommand\chaptername{BAB}
\AtBeginDocument{\renewcommand{\bibname}{DAFTAR PUSTAKA}}
\renewcommand\listfigurename{DAFTAR GAMBAR}
\renewcommand\listtablename{DAFTAR TABEL}
\renewcommand\figurename{Gambar}
\renewcommand\tablename{Tabel}
\renewcommand\appendixname{Lampiran}

% Setting numering di chapter dan section
\sloppy
\renewcommand\thechapter{\Roman{chapter}}
\renewcommand\thesection{\arabic{chapter}.\arabic{section}}
\renewcommand\thesubsection{\thesection.\arabic{subsection}}
\renewcommand\thesubsubsection{\thesubsection.\arabic{subsubsection}}
\renewcommand\thefigure{\arabic{chapter}.\arabic{figure}}
\renewcommand\thetable{\arabic{chapter}.\arabic{table}}
\renewcommand\theequation{\arabic{chapter}.\arabic{equation}}
\setcounter{secnumdepth}{3}

% Mengatur size dokumen
\titleformat{\chapter}[display]{\centering\normalfont\large\bfseries}{\chaptertitlename\ \thechapter}{100pt}{\large}
\titleformat{\section}{\normalfont\fontsize{12}{15}\bfseries}{\thesection}{1em}{}
\titleformat{\subsection}{\normalfont\fontsize{12}{15}\bfseries}{\thesection}{1em}{}
\titlespacing{name=\chapter,numberless}{0pt}{-30pt}{20pt}
\titlespacing{\section}{0pt}{10pt}{5pt}
\titlespacing{\subsection}{0pt}{10pt}{5pt}

% Library untuk daftar isi, gambar, dan tabel
\usepackage{tocloft}
\setlength{\cftbeforechapskip}{1pt} % Jarak sebelum bab
\setlength{\cftbeforesecskip}{1pt}  % Jarak sebelum subbab
\setlength{\cftbeforesubsecskip}{1pt} % Jarak sebelum sub-subbab
\renewcommand\cftdotsep{1}

% Pengaturan judul daftar isi, daftar tabel, dan daftar gambar
\renewcommand\cfttoctitlefont{\hfil\bfseries\normalsize} % Daftar Isi
\renewcommand\cftloftitlefont{\hfill\bfseries\normalsize} % Daftar Tabel
\renewcommand\cftlottitlefont{\hfill\bfseries\normalsize} % Daftar Gambar
\renewcommand\cftaftertoctitle{\hfil} % Posisi setelah judul Daftar Isi
\renewcommand\cftafterloftitle{\hfill} % Posisi setelah judul Daftar Tabel
\renewcommand\cftafterlottitle{\hfill} % Posisi setelah judul Daftar Gambar
\setlength{\cftbeforetoctitleskip}{0pt} % Jarak sebelum judul Daftar Isi
\setlength{\cftbeforeloftitleskip}{0pt} % Jarak sebelum judul Daftar Tabel
\setlength{\cftbeforelottitleskip}{0pt} % Jarak sebelum judul Daftar Gambar
\setlength{\cftaftertoctitleskip}{20pt} % Jarak setelah judul Daftar Isi
\setlength{\cftafterloftitleskip}{20pt} % Jarak setelah judul Daftar Tabel
\setlength{\cftafterlottitleskip}{20pt} % Jarak setelah judul Daftar Gambar

% Setting letak penomoran halaman awal chapter
\fancypagestyle{chapterfirstpage}{
    \renewcommand\headrulewidth{0pt}
    \fancyhf{} % Menghapus konfigurasi sebelumnya
    \fancyfoot[C]{\thepage} % Menetapkan nomor halaman di tengah bawah
}

% Setting letak penomoran halaman seterusnya
\fancypagestyle{otherpages}{
    \renewcommand\headrulewidth{0pt}
    \fancyhf{} % Menghapus konfigurasi sebelumnya
    \fancyhead[R]{\thepage} % Menetapkan nomor halaman di atas kanan
}

% penggunaan first line % gunakan \noindent untuk normal line
\setlength\parindent{36pt}

% mengilangkan garis header
\renewcommand\headrulewidth{0pt}

% pengaturan pemisahan kata
\pretolerance=10000 % Mengurangi pemisahan kata
\hyphenpenalty=10000 % Mengurangi pemisahan kata
